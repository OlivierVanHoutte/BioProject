\documentclass{esannV2}
\usepackage[dvips]{graphicx}
\usepackage[latin1]{inputenc}
\usepackage{amssymb,amsmath,array}
\usepackage{parskip}
\usepackage{multicol}

\voffset 0 cm \hoffset 0 cm \addtolength{\textwidth}{0cm}
\addtolength{\textheight}{0cm}\addtolength{\leftmargin}{0cm}


\begin{document}
%style file for ESANN manuscripts
\title{Bio Project}

\author{Stijn Rosaer$^1$ and Olivier Van Houtte$^2$
\vspace{.3cm}\\
1- Universiteit Antwerpen - Computer Science \\
Prinsstraat 13 2000 Antwerpen - Belgium
}
\maketitle

\begin{abstract}
Zoals we weten, is er interesse in het storen van data als DNA. Er bestaan al veel methoden om dit DNA op te bouwen en terug uit te lezen, maar de encodings wijze staat nog niet vast. In dit verslag achterhalen we ofdat er misschien tijd en/of plaats effici�ntere encoderings systemen zijn dan gewoon bit voor bit op te slaan en of deze in de praktijk toepasbaar zijn. 
\end{abstract}

\section{Inleiding}
Als we aan DNA denken, worden we al snel herinnerd aan de 4 basen. Adenine (A), guanine (G), cytosine (C) en thymine (T). Met deze basen kunnen we de simpelste vorm van encoding opstellen, namenlijk 2 bits per base. We kiezen dan voor elke base een unieke combinatie aan bits.
\newline

00 = A, 01 = G, 10 = C, 11 = T
\newline


We kunnen dan elke DNA string (met een even aantal bits) omzetten naar DNA.
\newline

0111001110 $\rightarrow$ GTATC
\newline

Het kan echter effici�nter. 


\section{Encoding}

In plaats van simpelweg 2 bit aan elke base toe te wijzen, stellen we nu eerst alle mogenlijke sets op bestaande uit unieke combinaties van de 4 basen. Er zijn er zo 15.

\begin{center}
\begin{multicols}{3}
	
\begin{enumerate}
\item A
\item G
\item C
\item T
\item A G
\item A C
\item A T
\item G C
\item G T
\item C T
\item A G C
\item A G T
\item A C T
\item G C T
\item A G C T
\end{enumerate}

\end{multicols}
\end{center}

We kunne  aan hier de getallen 0 t.e.m. 14 aan toewijzen. we kunnen deze zelfde getallen ook representeren met 4 bits.
Hierbij komen op een probleem. Met 4 bits kan je de 16 getallen representeren, namenlijk 0 t.e.m. 15. 1 Getal meer dan de 15 getallen die we met de base-sets bekomen. We zullen dus 1 bit moeten laten vallen. Dat brengt ons op 3 bits die de getallen 0 t.e.m. 7 voorstellen. Hierdoor kiezen we 8 sets in plaats van 15.
We geven ze ook telkens een nieuwe letter indien nodig, zodat we deze makkelijker in string vorm kunnen representeren. Hiervoor maken we gebruik van IUPAC standaard voor degenerate base symbols.

   
\begin{center}
\begin{tabular}{ c || c | c c |c}
	A & 0 & A && 000\\
	G & 1 & G && 001\\
	C & 2 & C && 010\\
	T & 3 & T && 011\\
	R & 4 & A & G& 100\\
	M & 5 & C & T& 101\\
	Y & 6 & A & C& 110\\
	K & 7 & G & T& 111\\
\end{tabular}
\end{center}

We kunnen nu weer een string van bits omzetten naar een string, gerepresenteerd de gekozen sets. Indien het aantal bits niet deelbaar is door 3, voegen we de nodige nullen toe aan het begin van de string waardoor elke base de representatie is van exact 3 bits.
\newline

(00)0111001110 $\rightarrow$ AKGY
\newline

Om de nodige DNA strings te verkrijgen, kiezen we voor elke letter een base uit de corresponderende set.
Waneer we een reeks data willen encoderen door DNA zullen we in dit geval, voor elke 3 bits, een corresponderende degenerate base vinden in de hierboven gedefini�erde tabel. Aangezien sommige bases een combinatie zijn van Adenine, Cytosine, Guanine en Thymine, zullen we meerdere DNA strings moeten genereren te encodering van een base op positie i in het resultaat op de overeenkomstige positie in elke string terug te vinden.
\newline
\begin{center}
\begin{tabular}{c}
	A K GCA Y \\
	\hline
	A$|$G$|$GCA$|$A$|$\\
	A$|$C$|$GCA$|$T$|$\\
	A$|$G$|$GCA$|$T$|$\\
	A$|$C$|$GCA$|$A$|$\\
	...\\
\end{tabular}
\end{center}

We bouwen nu zoveel strings als nodig is, wat zeer effici�nt kan met moderne technieken.
Wanneer we de data terug willen decoden, kiezen we weer een aantal DNA strings uit de gegenereerde strings.
We controleren welke basen we terug vinden, en achterhalen zo welke corresponderende letter hier mee overeen komt.
\newline

\subsection*{Voorbeeld}
 We zien op positie 3, in verschillende strings, de characters A en C. Dan weten we dat dit een Y representeert.

Stel dat er maar 1 soort base wordt gelezen op een bepaalde positie. Dan zijn er meerdere opties. 

\begin{itemize}
	\item We lezen inderdaad een letter met maar 1 base in zijn set. Correct.
	\item We lezen een letter met 2 basen in zijn set, maar zien maar 1 base in onze gekozen strings. Fout. 
\end{itemize}

Dit zorgt ervoor dat we niet altijd de correcte output terug krijgen. 

\section{Correctheid}
De kans op het correct lezen van een character (Denk ik):
\newline

$$1-\left(\frac{1}{2}\right)^{N}$$
\newline

\noindent
Dat maakt de kans op een volledig correct gelezen string
\newline

$$\left(1-\left(\frac{1}{2}\right)^{N}\right)^L$$
\newline

Met L het aantal characters in de DNA strings en N het aantal strings we gebruiken om te checken.

Onze tests komen zeer goed overeen met deze formule. 
We testen hier met een input DNA string van lengte L = 144;
\newline

\begin{tabular}{c|| c | c || c}
	N & tests ($\%$) & formule ($\%$) & verschil \\
	\hline
	4 & 0.0 & 0.0 & 0.0\\
	5 & 1.5 & 1.0  & 0.5\\
	6 & 12.3 & 10.4 & 1.9\\
	7 & 35.6 & 32.3 & 3.3\\
	8 & 58.6 & 57.0 & 1.6\\
	9 & 77.6 & 75.3 & 2.3\\
	10 & 88.3 & 86.9 & 1.4\\
	11 & 93.3 & 93.2 & 0.1\\
	12 & 97.2 & 96.5 & 0.7\\
	13 & 98.6 & 98.2 & 0.4\\
	14 & 99.3 & 99.1 & 0.2\\
	15 & 99.7 & 99.6 & 0.1\\
	16 & 99.8 & 99.8 & 0.0\\
	17 & 99.9 & 99.9  & 0.0\\
	18 & 99.9 & 99.9  & 0.0\\
	19 & 100 & 100  & 0.0\\
	...  & ... & ... & ...\\
	
	$\infty$ & 100 & 100  & 0.0\\
	
\end{tabular}
\newline

We verklaren de kleine verschillen met afrondings fouten.


\section{In de praktijk}

Het is simpeler om meerdere korte DNA strings te maken en lezen dan 1 lange. Dat is exact wat we met deze vorm van encoding doen. We maken i.p.v. 1 lange DNA string, veel korte, die we dan ook parallel kunnen inlezen indien nodig.
\newline

Zoals we weten bestaat DNA uit 2 strings. Zodat we zeker de juiste kant kiezen tijden het omzetten, kunnen we best elke DNA string beginnen en eindigen met een primer. Een kort stukje DNA van een vaste vorm, die het begin/ einde en de ori�ntatie van het DNA aanduidt. Dit verlengd onze DNA string een beetje, maar zou een relatief kleine impact hebben op het eindresultaat.

\section{Conclusie}

Er is dus zeker nog meer onderzoek mogenlijk naar DNA als data storage. Onze inspiratie voor deze paper was dan ook een artikel uitgebracht vorig jaar. Maar wij vinden dit al zeker een zeer goede encoding.

\begin{footnotesize}

\begin{thebibliography}{99}
	\bibitem{webArtikel} Choi, Y., Ryu, T., Lee, A.C. et al. High information capacity DNA-based data storage with augmented encoding characters using degenerate bases. Sci Rep 9, 6582 (2019). https://doi.org/10.1038/s41598-019-43105-w

\end{thebibliography}

\end{footnotesize}

\end{document}
